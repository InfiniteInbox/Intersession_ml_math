\documentclass[12pt, a4paper]{article}
\usepackage{graphicx}
\usepackage{listings}
\usepackage{color}
\usepackage{amsmath}
\usepackage{amssymb}


\title{An Exploration into Linear Algebra}
\author{Yash Money, Vince McMath}
\date{It's Too Early for This}

\definecolor{dkgreen}{rgb}{0,0.6,0}
\definecolor{gray}{rgb}{0.5,0.5,0.5}
\definecolor{mauve}{rgb}{0.58,0,0.82}

\lstset{frame=tb,
  language=Python,
  aboveskip=3mm,
  belowskip=3mm,
  showstringspaces=false,
  columns=flexible,
  basicstyle={\small\ttfamily},
  numbers=none,
  numberstyle=\tiny\color{gray},
  keywordstyle=\color{blue},
  commentstyle=\color{dkgreen},
  stringstyle=\color{mauve},
  breaklines=true,
  breakatwhitespace=true,
  tabsize=2,
  frame=tb
}

\begin{document}
\maketitle

\section{Basic Matrix Code}

To begin, we have will create code that represents matrices and
performs elementary matrix operations, such as computing the inverse, 
multiplying, and adding. 

We will utilise Python to do so. Python has a data structure known as a "list" or an "array."
These are essentially a collection of indexed data that can be manipulated. Lists may contain sublists; it is in this way
that we can represent a "matrix" in python, for indeed, a matrix is nothing but a collection of row/column vectors, which themselves
can be represented as an individual python list. 

We have decided to represent a matrix as a collection of row vectors.

For example consider the following matrix, $ A \in \mathbb{R}^{m\times n} $: 

\begin{equation*}
    A =
\begin{bmatrix}
a_{1}  & a_{2} & a_{3} & ... & a_{n}\\
b_{1}  & b_{2} & b_{3} & ... & b_{n}\\
...  & ...& ... & ... & ...\\
m_{1}  & m_{2} & m_{3} & ... & m_{n}\\
\end{bmatrix}
\end{equation*}

with $m$ rows and $n$ columns. We choose to represent the same matrix, $A$ pythonically in the following way: \newline

\begin{lstlisting}
array = [
    [a1, a2, a3, ..., aN],
    [b1, b2, b3, ..., bN],
    [c1, c2, c3, ..., cN],
    ...,
    [m, m2, m3, ..., mN],
]    
\end{lstlisting}

In this way, each sublist is a row of the matrix. 

In the following we, will discuss how to code basic matrix operations using this data structure.

\subsection{Multiplying a Matrix by a Scalar}

One of the most crucial elements of matrix arithmetic is the ability to multiply a given matrix by a scalar. 
When doing this, the following will hold true:

\begin{equation*}
    \forall \lambda \in \mathbb{R}, \forall A \in \mathbb{R}^{m\times n},  \lambda A = 
    \begin{bmatrix}
        \lambda a_{1}  & \lambda a_{2} &  \lambda a_{3} & ... & \lambda a_{n}\\
        \lambda b_{1}  & \lambda b_{2} & \lambda b_{3} & ... & \lambda b_{n}\\
        ...  & ...& ... & ... & ...\\
        \lambda m_{1}  & \lambda m_{2} & \lambda m_{3} & ... & \lambda m_{n}\\
    \end{bmatrix},
\end{equation*}

This is a rather simple problem to tackle. Simply, we iterate through each row of the matrix, and then through each
element (column) within that row, multiplying each entry by a given $\lambda$ as we iterate. 
This produces a matrix that has been multiplied by lambda. The code is as follows: \newline

\begin{lstlisting}
def matrix_by_scalar(matrix1, scalar_quantity):
    
    '''
    TAKES: 
        A matrix of the form outlined above, matrix1
        A scalar_quantity by which the matrix will be multiplied
    RETURNS:
        A matrix of the form outlined above
    '''
    try:
        if (isinstance(scalar_quantity, int)) or (isinstance(scalar_quantity, float)): 
        # O(1) we wcheck if the scalar element is a valid real number; if not, we raise an error.

            return  list([element * scalar_quantity for element in row] \
                for row in matrix1) # O(n**2) This comprehension multiplies each element of each row by the scalar and thus has a time complexity of n**2
    
        else: raise ValueError(f"Argument passed: '{scalar_quantity}'. Error: Expected argument of type 'int' or 'float' ") # error raised
    except: # In Case something goes wrong; biggest error here is the potential for incorrect args passed
        print("something went wrong - likely the matrix argument was incorrect") 

\end{lstlisting}

\subsection{Adding Two Matrices}

The addition of matrices is an elementwise one, and therefore it is easy to implemenent iteratively and makes lots of intuiative sense. 
What is meant by elementwise is that we can handle each element individually; in the case of matrix addition, 
we add each element of a given matrix to the corresponding element of a separate matrix. We can assert that the matrices must 
therefore be the of the same dimension to be added. 

Essentially, we have the following for $ \forall A, B \in \mathbb{R}^{m\times n}, $ where $i = 1,...,m$  and $j = 1,...,n$      

\begin{eqnarray*}
    A + B & = & \begin{bmatrix}
        A_{1,1}  & A_{1,2} & ... & A_{1,j}\\
        A_{2,1} & A_{2,2} & ... & A_{2,j}\\
        ... & ... & ... & ... \\
        A_{i,1} & A_{i,2} & ... & A_{i,j}
    \end{bmatrix} + \begin{bmatrix}
        B_{1,1}  & B_{1,2} & ... & B_{1,j}\\
        B_{2,1} & B_{2,2} & ... & B_{2,j}\\
        ... & ... & ... & ... \\
        B_{i,1} & B_{i,2} & ... & B_{i,j}
    \end{bmatrix} \\
    & = & \begin{bmatrix}
        A_{1,1} + B_{1,1} & A_{1,2} + B_{1,2} & ... & A_{1,j} + B_{1,j} \\
        A_{2,1} + B_{2,1} & A_{2,2} + B_{2,2} & ... & A_{2,j} + B_{2,j} \\
        ... & ... & ... & ... & \\
        A_{i,1} + B_{i,1} & A_{i,2} + B_{i,2} & ... & A_{i,j} + B_{i,j} \\
    \end{bmatrix}
\end{eqnarray*}

The code looks as follows:

\begin{lstlisting}
def add_matrices(mat1, mat2): 

    '''
    TAKES:
        two matrices of the standard form outlined previously, mat1 and mat2
            They must be of the same dimensions
    RETURNS:
        A single matrix of the same dimensions as mat1 and mat2, where this matrix is the sum of mat1 and mat2
    '''

    if (len(mat1) == (len(mat2)) and (len(mat1[0]) == len(mat2[0]))): # check to see if possible to add the two lists, as their dimensions must be the same

        '''
        The following is just a long list comprehension that iterates through each matrix and adds corresponding elements. It then appends these to "l", which is what is ultimately what is returned
        '''
        # basically just two for loops
        l = list([mat1[row][col]+mat2[row][col] \ 
            for col in range(len(mat1[0]))] \ 
            for row in range(len(mat1)))
        return l 
        # Intuitively subtraction is very similar: the "+" must be turned to a "-"

    else: # if the matrices are not of the same size.
        raise ValueError("Args are different size and thus cannot be added")

\end{lstlisting}

This code will hold for subtraction as well, because of the following identity:

\begin{eqnarray*}
    A - B & = & \begin{bmatrix}
        A_{1,1}  & A_{1,2} & ... & A_{1,j}\\
        A_{2,1} & A_{2,2} & ... & A_{2,j}\\
        ... & ... & ... & ... \\
        A_{i,1} & A_{i,2} & ... & A_{i,j}
    \end{bmatrix} - \begin{bmatrix}
        B_{1,1}  & B_{1,2} & ... & B_{1,j}\\
        B_{2,1} & B_{2,2} & ... & B_{2,j}\\
        ... & ... & ... & ... \\
        B_{i,1} & B_{i,2} & ... & B_{i,j}
    \end{bmatrix} \\
    & = & \begin{bmatrix}
        A_{1,1} - B_{1,1} & A_{1,2} - B_{1,2} & ... & A_{1,j} - B_{1,j} \\
        A_{2,1} - B_{2,1} & A_{2,2} - B_{2,2} & ... & A_{2,j} - B_{2,j} \\
        ... & ... & ... & ... & \\
        A_{i,1} - B_{i,1} & A_{i,2} - B_{i,2} & ... & A_{i,j} - B_{i,j} \\
    \end{bmatrix}
\end{eqnarray*}

where $ \forall A, B \in \mathbb{R}^{m\times n}, $ and $i = 1,..., m$, $j = 1,..., n$.

This implies that we must merely change the elementwise addition in our code to subtraction in order to create
a function that subracts two matrices.

\subsection{Transpose of a Matrix}

The tranpose of a matrix is a modification to its structure such that rows become columns and vice versa. That is, the first row of a given matrix
$A$ will be the first column of it's transpose, $A^{T}$, the second row of $A$ will be the second column of $A^{T}$, and so forth and so on.
\newline
\begin{eqnarray*}
    A &=& \begin{bmatrix}
        A_{1,1}  & A_{1,2} & ... & A_{1,j}\\
        A_{2,1} & A_{2,2} & ... & A_{2,j}\\
        ... & ... & ... & ... \\
        A_{i,1} & A_{i,2} & ... & A_{i,j}
    \end{bmatrix} \in \mathbb{R}^{m\times n}, i=1,...,\text{ and }j=1,...,n
    \\
    \\
    \text{Then,}
    \\
    \\
    A^{T} &=& \begin{bmatrix}
        A_{1,1}  & A_{2,1} & ... & A_{i,1}\\
        A_{1,2} & A_{2,2} & ... & A_{i,2}\\
        ... & ... & ... & ... \\
        A_{1,j} & A_{2,j} & ... & A_{i,j}
    \end{bmatrix} \in \mathbb{R}^{m\times n}, i=1,...,\text{ and }j=1,...,n
\end{eqnarray*}

In the case of square matrices, we can state the matrix has been "flipped" over it's main diagonal, 
or the diagonal elements running from top-left to bottom-right.
\newline
The transpose, as with matrix addition/subtraction and multiplying by a scalar, is rather trivial to implement. 
We iterate through the columns of a matrix, and return them to the user as rows of a new matrix, which is the tranpose. 
In regards to greabbing the columns of a matrix: python provides no way to index a column as simply as one can index a row. 
This is because, as we have represented matrices with row vectors, the columns are comprised of elements from mutliple different rows. 
Therefore, we construct a function that enables us to quickly "grab" a column from a matrix given its index. 

\begin{lstlisting}
def get_col(matrix_2d, _index):

    return list(row[_index] for row in matrix_2d) # O(n) this simply grabs the column from the specified index. 

    # full time O(n)

def transpose(matrix):

    new_array = [get_col(matrix, i) for i in range(len(matrix[0]))] # O(n) and nested O(n), becomes O(n**2). takes a column and makes it a row
    return new_array

\end{lstlisting}

\subsection{Row Reduction: Echelon/Upper Triangle Form}

One of the most important matrix-manipulations is the ability to perform elementary "row operations" upon a matrix. 
That is, multiplying a row by a scalar, swapping the position of two rows in a matrix, and adding/subtracting a scalar multiple of one row to/from another. 
These operations have incredibly useful properties, and are widely used to calculate determinants and inverses of matrices, and 
additionally are used to solved systems of linear equations using matrices. 

These row operations allow one to effectively mutate a matrix into one of many various forms; for example, we can row reduce such that
what we are left with is of the "upper triangle" form, in which all elements not above or in the main diagonal are 0. 

This form is known as "echelon" form, and the process to find it is called "Gaussian Elimination."

We begin the algorithm by swapping rows that need to be swapped. 

\begin{lstlisting}
def echelon(matrix): 
    
    # we will iterate through the columns
    for col_index in range(len(matrix[0])): 

        col = get_col(matrix, col_index) #we grab the column using the index from the above for loop

        '''
        the following bit of code looks for places where there might be zeroes in the diagonal.
        if there are, and we do not handle for it, we will get a divinding by zero error
        We then swap the rows such that we can proceed

        thus, the following code is quite necessary.  
        '''

        if col_index <= len(matrix): # O(1) we only need to look for zeroes in the first square - that is, if the matrix is longer than tall, it is uncessary to check all columns

            if all((i == 0) for i in col[col_index:]): #O(n) if the entire column is filled with zeroes, we call continue and the program returns to the initial for loop, and goes to the next column
                continue 
            
            elif col[col_index] == 0: # O(1) if one of the elements on the diagonal is zero - this is where the dividing by zero error occurs so we need to handle this
                ''' 
                here we iterate through all of the rows below the diagonal. 
                if we find a row that doesn't contain a zero in the diagonal column index, 
                we will swap them
                '''
                for i in range(len(col[col_index:])): #O(n) 
                    if col[col_index:][i] != 0: # O(1)
                        row_idx = col_index+i # O(1)
                        break 
                
                # the below line of code simple swaps the rows
                matrix[col_index], matrix[row_idx] = matrix[row_idx], matrix[col_index] # O(1)
\end{lstlisting}

Now that rows have been sufficiently swapped, we may begin subtracting scalar multiples of rows from rows below it. 
This enables us to achieve a zero in the desired spot. We will continue in the for loop that we currenly operate in. 

\begin{lstlisting}
    '''
        the following for loop is where the actual formula happens
        the algorithm works as follows:

        assume we have an array, 
        we have the nth column and we want to make all elements of that column below 
        the nth row into 0

        say n = 1 then we have:

        c1 = [           desired_c1 = [
              3                         3
              4                         4
              3                         0
              7                         0
              2                         0
                ]                         ]
        
        we can achieve this via subtracting a scalar multiple of the row such that we get 0
        ex: row 2. we can achieve row 1, col 2 equaling 0 by 
        subtracting row 1 * matrix[row1_idx][col2_idx]/matrix[row1_idx][row_idx]
        
        '''
        for row_index in range(len(col)): # O(n) we iterate through row of each colum we have grabbed earlier
            '''
            remember, we only want to turn the rows below
            the diagonal into 0. thus, we check if the row is indeed one we one to turn into 0
            if it is not, its idx will be less than the column idx
            if that proves to be true, we will simple pass
            '''

            if row_index <= col_index: #O(1)n checks if the row is one we do not want to turn to 0
                '''
                the following if statement is unnecessary, as I could have explicity called:
                    matrix[col_index][col_index] when I called denominator later
                    matrix[col_index] when I call raw_subtractant_row later
                    however, will keep this code for readability, as I find this easier to understand.
                '''
                if row_index == col_index: #O(1)
                    denominator = matrix[row_index][col_index] #O(1)
                    raw_subtractant_row = matrix[row_index] #O(1)
                pass              

            else:
                '''
                here we actually do the conversion to 0
                this finds teh numerator of the scalar we will multiply the subtractant row by
                then we will simply create the final subtractant row
                then we simply subtract the two rows, resulting in a 0
                we then replace the old row with the new one.                
                '''
                row_to_sub_from = matrix[row_index] # O(1)
                numerator = matrix[row_index][col_index] #O(1)

                subtractant = row_by_scalar(raw_subtractant_row, (numerator/denominator)) # O(n)
                subbed_row = subtract_row(row_to_sub_from, subtractant) # O(1)

                matrix[row_index] = subbed_row # O(1)
    
    return matrix

\end{lstlisting}

\subsection{Row Reduction: Row Echelon Form (REF)}

Consider a matrix that has been reduced such that it is in echelon form, where all elements below the main diagonal are 0. 
It is possible to reduce a matrix further. Specifically, by identifying all non-zero diagonal elements and dividing the rows in which
they appear by the element itself, we can get to a form similar to echelon form, but one where all non-zero diagonal elements will be 1. 
This is "row-echelon" form.

In other words, we can obtain the following, given $ A \in \mathbb{R}^{m\times n} $ and it is in echelon form:

\begin{eqnarray*}
    A & = & \begin{bmatrix}
        A_{1,1}  & A_{1,2} & ... & A_{1,j}\\
        0 & A_{2,2} & ... & A_{2,j}\\
        ... & ... & ... & ... \\
        0 & 0 & ... & A_{i,j}
    \end{bmatrix} \\     
    & \rightsquigarrow & \begin{bmatrix}
        1  & A_{1,2} & ... & A_{1,j}\\
        0 & 1 & ... & A_{2,j}\\
        ... & ... & ... & ... \\
        0 & 0 & ... & 1
    \end{bmatrix}
\end{eqnarray*}

It it is easy to turn a matrix in echelon form into one in row-echelon, as we must simply divide by the reciprocal of the diagonal element.

\subsection{Row-Reduced Echelon Form}

This is an extension of the REF form, where the intended result is 1 along the diagonal and zeros above and below the diagonal with constants on the rightmost column.
We do this by performing elementary row operations on a matrix from the row below to the row above to eliminate the constant terms with the exception of the diagonal. 
This method of elementary row operations
\begin{eqnarray*}
    A & = & \begin{bmatrix}
        1  & A_{1,2} & ... & A_{1,j}\\
        0 & 1 & ... & A_{2,j}\\
        ... & ... & ... & ... \\
        0 & 0 & ... & 1
    \end{bmatrix} \\     
    & \rightsquigarrow & \begin{bmatrix}
        1  & 0 & ... & 0\\
        0 & 1 & ... & 0\\
        ... & ... & ... & ... \\
        0 & 0 & ... & 1
    \end{bmatrix}
\end{eqnarray*}

\subsection{Finding Pivot Columns}

As we discussed earlier in the RREF (Reduced Row Echelon Form) section, our result from that operation
is a matrix taken from upper triangle form to one that has ones along its diagonal. In some cases, the matrix has linearly dependent columns,
in that case, we want to find the index of those columns for all types of matrices, no matter the size. We iterate through each row to find the first appearance of a 1
and check all prior elements to make sure they are 0's to account for non-square matrices.

\begin{lstlisting}
    def identify_pivots(mat, rounded=True):

    # takes matrix of form outlined in flowerbox
    # returns a list of all indexes in which there is a pivot column upon row reducing the input matrix

    # identifies the pivot columns of the matrix by row reducing

    matrix = mat.copy() # we make a copy because otherwise the original matrix is modified and we get fucked

    matrix = ref(matrix) # we run ref to row reduce it. we do not care about full row reduction because that it is trivial in this case
    
    if rounded: # this combats error where 0.99999999999 is not read as one, and therefore not read as a pivot
        matrix = mround(matrix, 12)

    pivot_col_idx_list = [] # we initialize a blank list will contain indexes of pivot cols

    for row in matrix: # we iterate through each row
        try: # we add a try in case there are no ones in the row
            first_one = row.index(1) # we find the first appearance of a 1 within the list
            for idx2 in range(first_one): # we then check if all prior elelements are 0
                if row[idx2] != 0: # if one is not a 0, we contiknue the loops
                    continue

            # however, if all elements prior to the first 1 are 0, we will append the index of that 1 to the pivcollist
            pivot_col_idx_list.append(first_one)

        except: # if there are no ones, it can be infered that there is no pivot column in that row, so we continue
            continue 

    return pivot_col_idx_list    

    # O(n**2) time
\end{lstlisting}


\subsection{Finding the Determinant}

\subsection{Inverting a Matrix}

\subsection{Multiplying Two Matrices}

\section{Vector(Sub)Spaces Pythonically}

\subsection{General Introduction to Vector Subspaces}

A vector space is special type of a group that contains both an inner and outer
operation, which are addition and multiplication by a scalar, respectively. Subspaces are, by proxy, spaces that retain the same properties and are a part of the original vector space

Early on, we realized that it was impossible to represent an infinite series of elements as a vector space in python, for example $\mathbb{R}^{2}$. 
We realized that we would have to have the user to insert a matrix representing that vector space. We created an umbrella class called "subspace".
These are integral to representing linear mappings pythonically.

Our subspace class is very broad, the operations done in the class are more of a general operations as stated below.

\subsection{Issues when Representing Subspaces Pythonically}

As I said earlier, we cannot express common vector spaces used in linear algebra and we had to settle for finite spaces. They would be in the form of:

\begin{eqnarray*}
    V & = &\begin{bmatrix}
        A_{1,1}  & A_{1,2} & ... & A_{1,j}\\
        A_{2,1} & A_{2,2} & ... & A_{2,j}\\
        ... & ... & ... & ... \\
        A_{j,1} & A_{j,2} & ... & A_{j,j}
    \end{bmatrix}
    \in \mathbb{R}^{j \times j}
\end{eqnarray*}

j represents the general dimensionality of the vector space. 

\subsection{Defining a Space with a Basis}

The definition of a basis is the set of linearly independent vectors that span a space. 
These have many useful applications when it comes to linear mapping. 

\begin{eqnarray*}
    U &=& span(\begin{bmatrix}
        0 \\
        1 \\
        2
    \end{bmatrix}, 
    \begin{bmatrix}
        1 \\
        1 \\
        2 
    \end{bmatrix},
    \begin{bmatrix}
        2 \\
        2 \\
        4
    \end{bmatrix})
\end{eqnarray*}

Basis Vector

\begin{eqnarray*}
    \begin{bmatrix}
        0 \\
        1 \\
        2
    \end{bmatrix}
\end{eqnarray*}

The code to do this involved taking the indexes of the pivot columns from earlier and then mapping those to the original matrix.
In the example above, the pivot columns would be a list with index [0] since the original matrix's RREF form comes out to be 
\begin{eqnarray}
    \begin{bmatrix}
        1 & 0 & 0 \\
        0 & 1 & 2 \\
        0 & 0 & 0
    \end{bmatrix}
\end{eqnarray}

\begin{lstlisting}
    def create_basis(self): 

        # we find the basis of generatign set

        tempgset = self.gset # we create a temporary gset to not modify the original

        if self.columned == False: # if the input generating vectors are row vecs
            tempgset = mp.transpose(tempgset) # we will make then col vecs

        holder = mp.identify_pivots(tempgset) # now we identify pivots of the generating set, O(n**2)

        return [self.gset[i] for i in holder] # then we will return all of the vectors
\end{lstlisting}

Then any valid subspace is just the span of the linearly independent basis vectors.

\subsection{Orthonormality of Basis Vectors}

Orthonormality of basis vectors is when basis vectors are the unit vectors (the distance of the vector is 1) and if it is orthogonal (purpendicular).
We can find this by taking the dot product, a type of inner product defined as $\begin{bmatrix} x & y \end{bmatrix}  \cdot  \begin{bmatrix} a & b \end{bmatrix} = xa + yb$. 
The following needs to be satisfied: \\
\\
            $\langle x \; , \; y \rangle = 0$
            \\
            $\langle x \; , \; x \rangle = 1$
\\
\\
Here is our dot product code:
\begin{lstlisting}
    def dot(v1, v2):
    # find the dot product of 2 vectors
    if not(isinstance(v1[0], list)) and not(isinstance(v2[0], list)) and (len(v1)==len(v2)):
        # if we find that the vectors are not 1 dimensional or that that they are different sizes
        total = 0
        for idx in range(len(v1)):
        # multiply elementwise
            total += v1[idx]*v2[idx]
        
        return total
\end{lstlisting}

These are very important to rotations of vectors. 
Here is our implemenentation of orthonormality:
\begin{lstlisting}
    def isorthonormal(self): 

        if self.columned == False: # if the input basis vecs are row vecs, a
            basis = mp.transpose(self.basis) # we will make them col vecs, but will use a new var to not modify self.basis
        else:
            basis = self.basis  # we don't want to fuck with self.basis

        if len(basis) >= 2: # orthonormal must have at least two basis vectors
            for idx, vec1 in enumerate(basis):  # now we run through each basis vector O(n)
                for vec2 in basis[idx+1:]: # we run through each other basis vecor
                    if (dot(vec1, vec2) != 0) or (dot(vec1, vec1) != 1): # if any of them do not pass the test (<v1,v2> = 0 or <v1,v1> =1) (dot defined lower)
                        return False # then we will return False because all must pass the test
                
            return True # otherwise, if False not returned yet, we know it to be true
        
        else: return False # if only one basis vec, nothing to be orthonormal to
\end{lstlisting}



\section{Representing Linear Mappings Pythonically}

As stated earlier, we would use the user inputted matrix representations of a vector space (generating set) and get a collection of basis vectors.
A linear mapping is a function that maps between 2 vector spaces and also prserves the following rules:
\\
\\
            $T: x \rightarrow y$
\\
\\
            $T(x + y) = T(x) + T(y)$
            \\
            $T(ax) = aT(x)$
\\
\\

where T is a function linearly mapping vector space x to y. In the case of a code application, these would be finite vector spaces.
To map a vector from one space to another, we need to find the transition matrix from the basis vectors we are converting with, which could even be 2 bases for each vector space. 
We will cover that code in the next section. Assuming we have the transition matrix, you can multiply that to the vector to recieve a vector that has been mapped to a new vector space. 
We can map a given vector with this code:

\begin{lstlisting}
    def map(self, vector):
        return self.apply(vector)

  def apply_mapping(self, vector):

    # expects a vector in the domain, not codomain
    # expects a vector in the domain, not codomain, of the form  vector = [element1, ...., element 2]
    # will then apply the linear mapping (self.mapping) to the vector
    # will return the result
    if len(vector) != self.domain_dim:
      return "hold up! wait a minute! sumn aint right"

    else:
      return mp.multiply_matrix(self.mapping, mp.transpose([vector]))

  def is_subspace(self, subspace):
      # check if subspace is a part of vector_space_1
      for vector in subspace:
          if vector not in self.matrix:
              return False
      return True
\end{lstlisting}

We can also check if a given vector space is a subspace by iterating through the original vector space and verifying if each element is in the vector space.

\subsection{Identifying The Type of Linear Mapping}

There are a handful of different ways to classify a mapping. These can tell us useful properties about a mapping:
\\
$\phi : J \rightarrow K $
\\ 
\\
Injective if $ \forall x,y \in J: \phi(x) = \phi(y)$
\\
Surjective if $ \phi(J) = K $
\\
Bijective if both hold
\\
\\
Homomorphism if its a linear mapping
\\
Isomorphism if $ \phi : J \rightarrow K$ linear and bijective
\\
Endomorphism if $ \phi : J \rightarrow J$ linear
\\
Automorphism if $ \phi : J \rightarrow J$ linear and bijective

Our code to do this is listed below:
\begin{lstlisting}
    class LinearMapping:

  def __init__(self, A, B, mapping=None):
    self.domain_basis = [A[i] for i in mp.identify_pivots(mp.transpose(A))]
    self.domain_dim = len(self.domain_basis)
    self.codomain_basis = [B[i] for i in mp.identify_pivots(mp.transpose(B))]
    self.codomain_dim = len(self.codomain_basis)
    self.domain = [row[:] for row in A]
    self.codomain = [row[:] for row in B]
    self.mapping = mapping # needs to be a matrix or None
    self.injective = True if mp.rank(self.mapping) == len(self.mapping[0]) else False
    self.surjective = True if mp.rank(self.mapping) == len(self.mapping) else False
    self.bijective = True if (self.injective == True and self.surjective == True) else False
    self.homomorphism = True # (by definition, duh)
    self.isomorphism = True if (self.bijective == True) else False # homomorphism already satisfied
    self.endomorphism  = True if len(A) == len(B) and len(A[0]) == len(B[0]) else False # True if dim(A) == dim(B) and len(A[0]) == len(B[0]) else False
    self.automorphism = True if self.endomorphism and self.bijective else False
\end{lstlisting}

\subsection{More advanced Basis Change}

To find the transformation matrix between basis vectors requires the RREF (Row-Reduced Echelon Form) of the new basis with the older basis vectors, we place the original basis vectors on the left and the desired basis on the right. 
We then take the right side of the augmented matrix; this can be extended to 2 pairs of bases as shown below:
\begin{lstlisting}
    def find_transition(og_base, new_base, columned=False):

    '''
    TAKES:
        original basis of subspace, og_base
        new basis of subspace, new_base

        it is important to pay mind to the shape of these basis vectors. 
        eg, input [[1,3,4], [1,4,7]] where each sublist is a basis vector will be treated as 3 col vectors, 
        [1,1], [3,4], and [4,7]. we added the columned flag to help with this
    '''

    if columned==False:
        og_base, new_base = transpose(og_base), transpose(new_base)

    # we find a transition matrix that changes the coordinates expressed from one base to another base

    # going from base 1 to base 2
    if len(og_base) == len(new_base) and len(og_base[0]) == len(new_base[0]):

        dim_to_take = len(og_base[0]) # we find what the dimensions of the transmat will be
        total = append_mat_right(new_base, og_base) # we create an augmented matrix with new basis on left and old basis on right
        reduced = rref(total) # row reduce, O(n**3)

        transition_matrix = list() # init our blank transition matrix

        for row in reduced: # we go through the reduced augmat to find what should be added to the transition matrix
            transition_matrix.append(row[-dim_to_take:])

        return transition_matrix
    
    # Full time O(n**3)
\end{lstlisting}
$ A'_{\phi} = T^{-1}A_{\phi}S $

Where $\phi : J \rightarrow K$ and ordered bases \\ 
$ B = (b_{1},...,b_{n})  B' = (b'_{1},...,b'_{n}) $
\\
$ C = (c_{1},...,c_{n})  C' = (c'_{1},...,c'_{n}) $

B is a tuple of the basis vectors in that given space and S represents the mapping of coordinate from B to B'.
T represents the mapping of coordinates from C to C' and $A_{\phi}$ is the Transformation matrix from B to C.
\begin{lstlisting}
    def change_transformation_basis(ogtransfmat, ogb1, ogb2, tildab1, tildab2, columned=False):
    
    '''
    TAKES:
        the transformation matrix/linmapping, ogtransfmat, of standard form outlined in flowerbox
        the original basis of domain, ogb1
        the original basis of codomain, ogb2
        the new basis of domain, tildab1
        the new basis of codomain, tildab2
        AS WITH find_transition, the base inputs can be finicky, the columned flag should help. check that for more indepth docs

    '''

    if columned==False: # we make sure the basis inputs are the correct form ie they are column vectors
        ogb1, ogb2, tildab1, tildab2 = transpose(ogb1),transpose(ogb2),transpose(tildab1),transpose(tildab2) 

    transition1 = find_transition(tildab1, ogb1) # we find transitino matrix from the new domain basis to the old domain basis, O(n**3)
    transition2 = inverse(find_transition(tildab2, ogb2)) # we find inverse of transition matrix from new codomain basis to old codomain basis, O(n**3)

    return multiply_matrix(multiply_matrix(transition2, ogtransfmat), transition1) # we multiply the matrices according to formula, O(n**3)

    # full time O(n**3)
\end{lstlisting}



\section{More Advanced Matrix Operations}

\subsection{Solving Homogeneous System}

Some matrices contain linearly dependent columns or rows making certain variables "free". We learned about the -1 trick for solving such matrices.
\begin{lstlisting}
    def solve_homogeneous(coef_matrix):
    # takes matrix of form outlined in flowerbox
    # returns a matrix using the minus 1 trick to solve homogeneous linear systems

    coef_matrix = rref(coef_matrix) # takes the rref of the coefficient matrix (system of linear eq) O(n**3)

    pivotcols = identify_pivots(coef_matrix) # finds the pivot columns of the rref matrix O(n**2)
    notpiv_cols = list() # since this function is called because of different size dimension matricies, there are going to be necessary added piv-columns
    for i in range(len(coef_matrix[0])): # iterates through the columns O(n)
        if i not in pivotcols: # Iterates through the list of pivot columns
            notpiv_cols.append(i) # appends the non pivot column to the non-pivot col list
    
    if notpiv_cols == []: # if the non-pivot columns are empty (in other words had total rank)
        return [0 for i in range(len(coef_matrix[0]))]

    if len(coef_matrix) == len(coef_matrix[0]):

        for idx in notpiv_cols:
            coef_matrix[idx][idx] = -1
        coef_matrix = transpose(coef_matrix)
        final = [coef_matrix[idx] for idx in notpiv_cols]
        return final


    for idx in notpiv_cols: #Iterate through the non pivot columns
        coef_matrix.insert(idx, [0 if i != idx else -1 for i in range(len(coef_matrix[0]))]) # by the minus 1 trick, we insert a row with minus 1 in the nth index to preserve diagonality
    coef_matrix = transpose(coef_matrix) # we transpose the matrix
    final = [coef_matrix[idx] for idx in notpiv_cols] # we return the solution as a square matrix  
    return final

    # O(n**3)
\end{lstlisting}


\section{Projections}

\subsection{Solving Homogeneous Systems}

\subsection{Finding Eigenvalues and Eigenvectors Of A Matrix}

\subsection{Finding the Eiegendecomposition}

\subsection{Finding the Singular Value Decomposition}

\end{document}