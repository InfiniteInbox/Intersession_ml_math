\documentclass[12pt, a4paper]{article}
\usepackage{graphicx}
\usepackage{listings}
\usepackage{color}
\usepackage{amsmath}
\usepackage{amssymb}


\title{An Exploration into Linear Algebra}
\author{Yash Money, Imran Iftikar}
\date{Idk man, I just work here}

\definecolor{dkgreen}{rgb}{0,0.6,0}
\definecolor{gray}{rgb}{0.5,0.5,0.5}
\definecolor{mauve}{rgb}{0.58,0,0.82}

\lstset{frame=tb,
  language=Python,
  aboveskip=3mm,
  belowskip=3mm,
  showstringspaces=false,
  columns=flexible,
  basicstyle={\small\ttfamily},
  numbers=none,
  numberstyle=\tiny\color{gray},
  keywordstyle=\color{blue},
  commentstyle=\color{dkgreen},
  stringstyle=\color{mauve},
  breaklines=true,
  breakatwhitespace=true,
  tabsize=2,
  frame=tb
}

\begin{document}
\maketitle

\section{Basic Matrix Code}

To begin, we have will create code that represents matrices and
performs elementary matrix operations, such as computing the inverse, 
multiplying, and adding. 

We will utilise python to do so. Python has a data structure known as a "list" or an "array."
These are essentially a collection of indexed data than can be manipulated. Lists may contain sublists; it is in this way
that we can represent a "matrix" in python, for indeed, a matrix is nothing but a collection of row/column vectors, which themselves
can be represented as an individual python lists. 

We have decided to represent a matrix as a collection of row vectors.

For example consider the following matrix, $ A \in \mathbb{R}^{m\times n} $: 

\begin{equation*}
    A =
\begin{bmatrix}
a_{1}  & a_{2} & a_{3} & ... & a_{n}\\
b_{1}  & b_{2} & b_{3} & ... & b_{n}\\
...  & ...& ... & ... & ...\\
m_{1}  & m_{2} & m_{3} & ... & m_{n}\\
\end{bmatrix}
\end{equation*}

with $m$ rows and $n$ columns. We choose to represent the same matrix, $A$ pythonically in the following way:

\begin{lstlisting}


    
\end{lstlisting}

\subsection{Matrix Multiplication}

\begin{lstlisting}
    def project_vector(vector, proj_mat, return_error=False):

    '''
    if using sub_space obj, will look like newvec = project_vector(vec, sub_space.proj_mat)
    '''
    
    projected = mp.multiply_matrix(proj_mat, [vector])

    if return_error == False:
        return projected[0]
    else:
        return mp.mround(projected)[0], euclidean_norm(mp.subtract_row(vector, projected[0]))
\end{lstlisting}

\section{Representing Subspaces Pythonically}

\section{Representing Linear Mappings Pythonically}

\section{More Advanced Matrix Operations}

\subsection{Solving Homogeneous Matricies}

\subsection{Finding Eigenvalues and Eigenvectors Of A Matrix}

\subsection{Finding the Eiegendecomposition}

\subsection{Finding the Singular Value Decomposition}

\end{document}